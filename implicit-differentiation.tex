\documentclass[a4paper,12pt]{article}
\usepackage{amsmath}

\title{Calculus 1}
\author{Emanuel Lugo Rivera}
\date{\today}
\begin{document}

\section*{Explicit Function}
If a relatoion between two variables is written in such a way that one variable alone $(exponent = 1)$ is on one side of the equality sign
and on the other side is a function of the other variable, like in $y= f(x)$, then $y$ is said to be an \textbf{\textit{explicit function}} of x.

\subsection*{Examples:}
In $y = 3x^2 -7, y$ has been  \textit{expressed explicitly} as a function of x.
In $u = sin t, u$ has been expressed \textit{expressed explicitly} as a function of t.


\end{document}