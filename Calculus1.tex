\documentclass[a4paper,12pt]{article}
\usepackage{amsmath}
\usepackage{parskip}
\usepackage{graphicx}
\graphicspath{"D:Projects/latek/graph-desmos.png"}

\title{Calculus 1}
\author{Emanuel Lugo Rivera}
\date{\today}
\begin{document}
\begin{titlepage}
    \maketitle
\end{titlepage}


\section{Limites y Continuidad}
\subsection{El limite de una constante es la constante misma}
Cuando trabajamos con limites cuando se acercan hacia un entero, vemos este comportamiento.
$$\displaystyle \lim_{k\to a}= K$$
\subsection{Cuando trabajamos con limites hacia $x=a$}
Se demuestra el procedimiento de resolver esta proxima expresion.
$$\displaystyle \lim_{x\to-5} f(x)= \frac{x+5}{25-x^2}$$
$$\displaystyle\lim_{x\to-5} f(x)=\frac{x+5}{(5+x)(5-x)}$$
$$\displaystyle \lim_{x\to-5} f(x)=\frac{1}{x-5}$$
$$\displaystyle f(x)=\frac{1}{10}$$
\section{Limite hacia Infinito}
Como podemos ver en la grafica de $f(x)=\frac{1}{x}$


\section{Derivada}
\subsection{Definicion de Derivada}
Aqui estaremos demostrando las definiciones de derivadas. La definicion de derivada es simplemente buscar la pendiente en una recta tangente.
Se puede demostrar con esta siguiente expresion:
$$\lim_{h \to 0} f(x) = \frac{f(x+h)-f(x)}{h}$$

\section{Reglas}
\subsection{Coeficientes}
Como podemos ver del ejemplo anterior la derivada es el limite de la recta tangente.
Como ya sabemos como trabajar con una funcion normal, veremos a ver como podemos trabajar derivadas con coeficientes.
$$(c(x))'= c'(x)$$

\subsection{Suma y Resta}
Aqui estaremos demostrando la formula para poder resolver una suma o resta.
$$(f(x)+g(x))' = f'(x)+g'(x)$$
$$(f(x)-g(x))'= f'(x)-g'(x)$$
\subsection{Producto}
La regla del producto es similar a la de suma y resta solo que conlleva mas pasos a la hora de hacer los computos
$$(f(x)g(x))'=f(x)g'(x)+f'(x)g(x)$$
\subsection{Regla del Cociente}
A diferencia de la regla del producto aqui estaremos haciendo lo opuesto.
$$\frac{f(x)'}{g(x)'} = \frac{g(x)f'(x)-f(x)g'(x)}{(g(x))^2}$$


\end{document}